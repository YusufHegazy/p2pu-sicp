% (c) Andres Raba 2011
% GPL v.3

\documentclass{article}

%\usepackage[scaled=0.85]{beramono}
%\usepackage{fourier}
\usepackage{inconsolata}
\usepackage[bitstream-charter]{mathdesign}

\usepackage{vmargin}
\setpapersize{custom}{130mm}{170mm}
\setmarginsrb{1.5cm}{1.0cm} % left, top
             {1.5cm}{1.0cm} % right, bottom
             {12pt}{10pt} % head height, sep
             {12pt}{10pt} % foot height, sep
\usepackage{fancyhdr}
\pagestyle{fancy}
\lhead{}
\chead{}
\rhead{\thepage}
\lfoot{}
\cfoot{}
\rfoot{}
\renewcommand{\headrulewidth}{0pt}
\renewcommand{\footrulewidth}{0pt}
\usepackage{amsmath}
\newenvironment{rcase}{\left.\begin{aligned}}{\end{aligned}\right\rbrace}
\usepackage{setspace}
\onehalfspacing
\usepackage{fancyvrb}

\newcommand{\D}{\displaystyle}
\newcommand{\T}{\textstyle}

% --------------------------------------------------
\usepackage{titlesec}
\usepackage[usenames,dvipsnames]{color}

\definecolor{pythonblue}{RGB}{54,112,160}
\definecolor{gray}{RGB}{112,112,112}

\titleformat{\section}
{\color{pythonblue}\normalfont\huge\bfseries}
{}{}{{\color{pythonblue}}}

\titleformat{\subsection}
{\color{gray}\normalfont\Large\bfseries}
{}{}{{\color{gray}}}
% --------------------------------------------------

\begin{document}

\section*{Exercise 1.13}

\subsection*{Golden ratio and Fibonacci numbers}

Using the definition of Fibonacci numbers:
$$ 
{\rm Fib}(n) = 
\begin{cases}
	0 & \text{if $n = 0$} \\
	1 & \text{if $n = 1$} \\
	{\rm Fib}(n-1) + {\rm Fib}(n-2) $\quad$ & \text{otherwise}
\end{cases}
$$
and mathematical induction, we will prove that
$$ \text{Fib}(n) = \frac{\varphi^n - \psi^n}{\sqrt{5}}. $$

%\subsection*{Proof}

First, we take $n = 0$ and $n = 1$ as induction base and show that 
$ \text{Fib}(n) = \D\frac{\varphi^n - \psi^n}{\sqrt{5}} $ is valid in these cases.
$$
\begin{array}{rcl}
\text{Fib}(0) & = & \D\frac{\varphi^0 - \psi^0}{\sqrt{5}} = 
  \D\frac{1 - 1}{\sqrt{5}} = 0 \\

\hspace{1ex} & & \\

\text{Fib}(1) & = & \D\frac{\varphi^1 - \psi^1}{\sqrt{5}} =
  \D\frac{1}{\sqrt{5}} \left(\D\frac{1 + \sqrt{5}}{2} - 
\D\frac{1 - \sqrt{5}}{2}\right) =
  \D\frac{2\sqrt{5}}{2\sqrt{5}} = 1 
%\end{displaystyle}
\end{array}
$$
Yes, they agree with the definition. 

Next, we presume that the following is true for some $k < n$:
$$ \text{Fib}(k) = \frac{\varphi^k - \psi^k}{\sqrt{5}}. $$
We will show that the truth of last statement implies the truth of 
$$ \text{Fib}(k + 1) = \frac{\varphi^{k+1} - \psi^{k+1}}{\sqrt{5}}. $$

By definition of the Fibonacci sequence, and using the equations 
$\varphi^2 = \varphi + 1$ and $\psi^2 = \psi + 1$ we have:
\begin{eqnarray*}
\text{Fib}(k+1) & = & \text{Fib}(k) + \text{Fib}(k-1) =
\frac{\varphi^k - \psi^k}{\sqrt{5}} + 
\frac{\varphi^{k-1} - \psi^{k-1}}{\sqrt{5}} \\
& = &
\frac{\varphi^{k-1} (\varphi + 1) - \psi^{k-1} (\psi + 1)}{\sqrt{5}} = 
\frac{\varphi^{k-1} \varphi^2 - \psi^{k-1} \psi^2}{\sqrt{5}} \\
& = &
\frac{\varphi^{k+1} - \psi^{k+1}}{\sqrt{5}}. \quad \text{QED.}
\end{eqnarray*}

We have just established that the induction step is valid. This can now be
used to show that if a statement with $n = k$ is true, then the one with 
$n = k + 1$ is also true. We have already demonstrated that the base cases
with $n = 0$ and $n = 1$ are true. These imply that the case with $n = 2$ 
must also be true. Step by step, this leads to any $n$. Therefore, we can 
safely assert the truth in general, with all $n$.

\end{document}
