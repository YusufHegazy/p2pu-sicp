% (c) Andres Raba 2011
% GPL v.3

\documentclass{article}

%\usepackage[scaled=0.85]{beramono}
%\usepackage{fourier}
\usepackage{inconsolata}
\usepackage[bitstream-charter]{mathdesign}

\usepackage{vmargin}
\setpapersize{custom}{130mm}{170mm}
\setmarginsrb{1.5cm}{1.0cm} % left, top
             {1.5cm}{1.0cm} % right, bottom
             {12pt}{10pt} % head height, sep
             {12pt}{10pt} % foot height, sep
\usepackage{fancyhdr}
\pagestyle{fancy}
\lhead{}
\chead{}
\rhead{\thepage}
\lfoot{}
\cfoot{}
\rfoot{}
\renewcommand{\headrulewidth}{0pt}
\renewcommand{\footrulewidth}{0pt}
\usepackage{amsmath}
%\usepackage{setspace}
%\onehalfspacing
\usepackage{fancyvrb}

% --------------------------------------------------
\usepackage{titlesec}
\usepackage[usenames,dvipsnames]{color}

\definecolor{pythonblue}{RGB}{54,112,160}
\definecolor{gray}{RGB}{112,112,112}

\titleformat{\section}
{\color{pythonblue}\normalfont\huge\bfseries}
{}{}{{\color{pythonblue}}}

\titleformat{\subsection}
{\color{gray}\normalfont\Large\bfseries}
{}{}{{\color{gray}}}
% --------------------------------------------------

\begin{document}

\section*{Exercise 1.19}

\subsection*{Derivation of the new transformation}

Here is the general transformation on a pair of integers:
$$  T_{pq}(a, b) = 
\begin{cases} 
    a \gets a(p + q) + bq \\
    b \gets bp + aq
\end{cases} $$
When $p = 0$ and $q = 1$, this reduces to the familiar case generating 
consecutive \emph{Fibonacci} numbers if used with the seed pair $(1, 0)$:
$$ T = T_{01}(a, b) = 
\begin{cases} 
    a \gets a + b \\
    b \gets a
\end{cases} $$
We get a new transformation $T_{p'q'}$ by applying the transformation $T_{pq}$ 
twice:
$$ T_{p'q'} = T_{pq}(T_{pq}) = T_{pq} \cdot T_{pq} = T_{pq}^2 $$
This means that we replace every $a$ with $a(p + q) + bq$ and every $b$ 
with $bp + aq$ in the first formula of $T_{pq}$: 
$$  T_{pq}^2(a, b) = 
\begin{cases} 
    a \gets (a(p + q) + bq)(p + q) + (bp + aq)q \\
    b \gets (bp + aq)p + (a(p + q) + bq)q
\end{cases} $$
After multiplying out and rearranging, we get:
$$  T_{pq}^2(a, b) = 
\begin{cases} 
    a \gets a(p^2 + q^2 + 2pq + q^2) + b(2pq + q^2) \\
    b \gets b(p^2 + q^2) + a(2pq + q^2)
\end{cases} $$
By comparing this to the first formula, we see that:
$$ \begin{cases} 
  p' = p^2 + q^2 \\
  q' = 2pq + q^2
\end{cases} $$
Now we are ready to use this result to complete the program.

\newpage
\subsection*{Program}

\begin{Verbatim}
(define (square x) (* x x))

(define (fib n)
  (fib-iter 1 0 0 1 n))

(define (fib-iter a b p q count)
  (cond ((= count 0) b)
	((even? count)
	 (fib-iter a
		   b
		   (+ (square p) (square q)) ; p'
		   (+ (* 2 p q)  (square q)) ; q'
		   (/ count 2)))
	(else (fib-iter (+ (* b q) (* a q) (* a p))
			(+ (* b p) (* a q))
			p
			q
			(- count 1)))))
\end{Verbatim}

\end{document}
